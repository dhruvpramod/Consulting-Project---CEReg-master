from:	Patrick Boily <pboily@uottawa.ca>
to:	Margaret Skwara <Margaret.Skwara@neb-one.gc.ca>
cc:	Jose Ribas Fernandes <Jose.ribasfernandes@cer-rec.gc.ca>,
Dhruv Pramod <dpram078@uottawa.ca>,
Andrew Willits <awill165@uottawa.ca>,
Smit Patel <spate128@uottawa.ca>,
Maia Pelletier <mpell074@uottawa.ca>
date:	Nov 1, 2019, 10:04 AM
subject:	RE: October Progress Report - Construction Incidents on Canadian Pipelines


Dear Ms. Skwara, how do you do?

 

As detailed in our proposal, we will be sending two monthly progress reports. In the October report, we outline the general tasks that have been completed-to-date, discuss current tasks being worked on, and provide a forecast of the work that needs to be completed.

 

Completed Tasks:
- Incident data has been collected and “cleaned”
- PDF scraper has been outlined and has undergone various testing stages

 

Current Tasks:
- Hard coding of the PDF scraper
- Preliminary analysis of incident data

 

Next Steps:
- Structuring and analysis of PDF data. Due to lack of consistency in the completeness of the various incident reports, we will require domain expertise from CEReg.  
- Building for final analysis.
- Designing and structuring presentation and reporting (dashboard).

 

Questions and Clarifications:
- What data/knowledge is specifically of interest to be extracted from the PDFs?
- What are the major points that would be of interest for reporting (i.e. in a dashboard)?We appreciate your continued collaboration and interest in this project.

 

Would you be available for a short conference call next week?

 

Regards

 

Patrick Boily, Smit Patel, Maia Pelletier, Dhruv Pramod, Andrew Willits  